% This file was generated with po4a. Translate the source file.
%
\documentclass[11pt,a4paper,sans]{moderncv}
\moderncvstyle{classic}

\usepackage[scale=0.80]{geometry}

\makeatletter

\moderncvcolor{blue}

\nopagenumbers{}

\firstname{Lluís} \familyname{Ramon Callao}
\title{Data Scientist}

\address{Melcior de Palau 25}{Barcelona 08028} \mobile{609236313}
\email{lluisramon.tgn@gmail.com}
\social[linkedin][www.linkedin.com/in/lluisramon]{Lluís Ramon}
\social[github][www.github.com/LluisRamon]{LluisRamon}
\social[twitter][https://twitter.com/lluisr_tgn]{@LluisR\_Tgn} 

\photo[75pt][0.2pt]{../Fotos/Lluis_pedra_dni.jpg}


\begin{document}

\makecvtitle

\section{Educació}

\cventry{2013 - 2016}{Màster en Estadística i Investigació
Operativa}{Universitat Politècnica de Catalunya - Universitat de
Barcelona}{Barcelona}{}{Master Thesis: Using tree-based models to predict
drunk driving at Police preventive checkpoints.}

\cventry{2002 - 2007}{Llicenciatura de Matemàtiques}{Universitat Politècnica
de Catalunya}{Barcelona}{}{}

\cventry{2007}{Becari del programa Socrates-Erasmus}{Université Pierre et
Marie CURIE}{París}{}{}

\section{Experiència}

\cventry{2016 - Present}{Data Scientist}{Digital Origin}{Barcelona}{}{
Digital Origin és una Fintech dedicada al finançament al consum amb
productes online, automàtics i sense papers. \newline La meva
responsabilitat principal és millorar el Credit Risk Engine per reduir el
risk i aumentar l'acceptació de clients. \newline Les principals tasques
realitzades són:
\begin{itemize} \item Crear models de Machine Learning i noves regles de
negoci
\begin{itemize} \item Definició, obtenció i manipulació de dades, feature
engineering
\item Feature selection, model training i validació
\item Col$\mathord{\cdot}$laborar amb l’equip de desenvolupament per posar
el model en producció
\item Seguiment i validació del model a producció
\end{itemize}
\item Optimització del portfoli per millorar el trade-off entre acceptació i
risc
\item Dissenyar un nou flux de returnings per millorar els KPI associats
\end{itemize} }

\cventry{2007 - 2016}{Tècnic en seguretat viària i mobilitat}{Servei Català
de Trànsit, Generalitat de Catalunya}{Barcelona}{}{
El Servei Català de Trànsit és l’autoritat responsable de la seguretat
viària i la mobilitat a Catalunya \newline
La meva responsabilitat principal era el seguiment dels KPI de seguretat
viària i de mobilitat a Catalunya. \newline
Les principals tasques realitzades eren:
\begin{itemize}
\item  Automatitzar l’anàlisi de dades, mapes i reporting amb llibreries
internes d`R
\item Validació de dades d’accidents, controls alcoholèmia, radar, espires
electromagnètiques, sancions.
\item Disseny d`estudis, recollida de dades i la seva posterior anàlisi
\item Definició dels principals KPI per al correcte seguiment de la
seguretat viària
\item Col$\mathord{\cdot}$laborar en la definició del Pla de seguretat
viària de Catalunya
\item Participar en projectes europeus i realitzar propostes a la Comissió
Europea
\item Enllaç amb proveïdors, realització de plecs i seguiment de les
consultores externes
\end{itemize}}

\cventry{Jun. 2015}{Curs d’introducció al llenguatge de programació
R}{DatoSphera}{Barcelona}{}{Formació de 16 hores a l’equip tècnic de
DatoSphera. El curs va incloure una introducció als elements bàsics del
llenguatge R, com programar eficientment, tractament de dades, connexió a
base de dades Elastic Search, gràfics d’alta qualitat, mapes, així com
generar informes automàtics.}{}

\pagebreak

\section{Idiomes}

\cvitem{Català}{Competència nativa} \cvitem{Castellà}{Competència nativa}
\cvitem{Anglès}{Competència professional completa}
\cvitem{Francès}{Competència bàsica limitada}

\section{Aptituds i Coneixements}

\cvitem{Programació}{SQL, MongoDB, Python, TEX, Git.} \cvitem{Data
Analysis}{R, Matlab, SPSS, SAS.} \cvitem{OS}{Windows, Mac, Linux.}
\cvitem{BI}{Microstrategy, Tableau.}

\section{Cursos}

\cventry{Apr. 2012}{Programming a Robotic Car}{Udacity}{Online}{}{}
\cventry{Mar. 2011}{Diseny i tractament d’enquestes pel mostreig}{Institut
d'Estadística de Catalunya}{Barcelona}{}{} \cventry{Sep. 2010}{Curs de
contractes del sector públic,}{Escola d'Administració Pública de
Catalunya}{Online}{}{} \cventry{Sep. 2009}{Agata GIS}{Servei Català de
Trànsit}{Barcelona}{}{} \cventry{May 2009}{Microstrategy
Desktop}{Microstrategy Business Intelligence}{Barcelona}{}{}
\cventry{Apr. 2008}{Aimsun Foundation Course}{Transport Simulation
Systems}{Barcelona}{}{} \cventry{Feb. 2008}{Certificat d’Aptitut
Pedagògica}{Universitat Politècnica de Catalunya}{Barcelona}{}{}

\section{Publicacions}

\cventry{Oct. 2016}{\httplink[Drinking patterns and drunk-driving behavior
in Catalonia, Spain: A comparative
study]{https://www.sciencedirect.com/science/article/abs/pii/S1369847816303709}}{Manuela
Alcañiz, Miguel Santolino, Lluís Ramon}{Transportation Research Part F:
Traffic Psychology and Behavior}{}{}

\cventry{Apr. 2014}{\httplink[Prevalence of alcohol-impaired drivers based
on random breath tests in a roadside survey in Catalonia
(Spain)]{www.sciencedirect.com/science/article/pii/S0001457514000037}}{Manuela
Alcañiz, Montserrat Guillén, Miguel Santolino, Daniel Sánchez-Moscona, Oscar
Llatje, Lluís Ramon}{Accident Analysis \& Prevention}{}{}

\section{Presentacions}

\cventry{Jan. 2016}{\link[Setting Rstudio Server using Amazon Web
Services]{http://lluisramon.github.io/RStudio-Server-AWS/Presentacion.html}}{Lluís
Ramon}{Barcelona R user's group}{Barcelona}{}{}
\cventry{Nov. 2015}{\link[Step by step introduction to
R]{http://lluisramon.github.io/intro_R_paso_a_paso/introducion_paso_a_paso.html}}{Lluís
Ramon}{Barcelona R user's group}{Barcelona}{}{}
\cventry{Jul. 2015}{\link[Package development
tutorial]{http://lluisramon.github.io/Taller_packages_R/Taller_packages_R}}{Lluís
Ramon}{Barcelona R user's group}{Barcelona}{}{}
\cventry{Feb. 2015}{\link[Automatic Reporting with
rmarkdown]{http://lluisramon.github.io/rmarkdown-talk/Presentacio}}{Carlos
Bort \& Lluís Ramon}{Barcelona R user's group}{Barcelona}{}{}
\cventry{Apr. 2014}{\link[Introduction to maps with
R]{http://lluisramon.github.io/Introduccio-mapes-amb-R/Intro_mapes_amb_R.html}}{Lluís
Ramon}{Barcelona R user's group}{Barcelona}{}{}
\cventry{Feb. 2014}{\link[Easy Dates and Times in
R]{http://rpubs.com/Lluis_Ramon/Easy_Date_Times_R}}{Lluís Ramon}{Barcelona R
user's group}{Barcelona}{}{} \cventry{Apr. 2013}{Easy package development
with RStudio}{Lluís Ramon}{Barcelona R user's group}{Barcelona}{}{}
\cventry{Nov. 2012}{\link[A practical introduction to ggplot2 and
ggmap]{http://rpubs.com/Lluis_Ramon/Prestantacion_ggplot2_ggmap}}{Andreu
Vall, Roger Borras, Lluís Ramon}{IV Spanish R User
Conference}{Barcelona}{}{} \cventry{Oct. 2012}{\link[Introduction to maps
with R]{http://rpubs.com/Lluis_Ramon/Presentacio_mapes_amb_R}}{Lluís
Ramon}{Barcelona R user's group}{Barcelona}{}{}
\cventry{Jun. 2012}{Introduction to R}{Lluís Ramon}{Barcelona R user's
group}{Barcelona}{}{} \cventry{Feb. 2012}{Data transformations with R}{Lluís
Ramon}{Barcelona R user's group}{Barcelona}{}{} \cventry{Oct. 2009}{SIDAT: A
comprehensive system for collecting road accidents data}{Lluís Ramon}{1st
Meeting on Road Safety for Cities}{Gijón}{}{}

\section{Premis}

\cventry{Oct. 2018}{eDreams ODIGEO Datathon - Second Best Accuracy
Model}{Predir el tipus d’allotjament (hotel, alberg, pensió, apartament)
utilitzant diverses imatges i etiquetes de Google Cloud Vision. Equip amb
David Massip i Jordi Puigdellivol}{}{}{}{}

\cventry{Oct. 2016}{Social Point Monster Legends Datathon - Best Business
Insights Track}{Millor Business Insights Track i tercer millor sistema de
recomanacions sobre dades d’un videojoc. Equip amb Jordi Zamora, Carlos Bort
i Jordi Puigdellivol}{}{}{}{}

\cventry{May 2016}{\link[Accenture Datathon 2016 - Best Accuracy Track
Model]{https://www.accenture.com/es-es/careers/analytics-datathon}}{Predicció
d’accidents de trànsit a Barcelona a partir d’informes d’accidents policials
locals. Equip amb Carlos Bort, Jordi Puigdellivol i Jordi Zamora}{}{}{}{}

\cventry{Nov. 2015}{\link[Social Point Dragon City Datathon - Best Accuracy
Track Model]{http://bcnanalytics.com/hackathon/}}{Millor model predictiu per
detectar jugadors que es donaran de baixa a un videojoc. Equip amb Carlos
Bort, Jordi Puigdellivol i Jordi Zamora}{}{}{}{}

\end{document}
