% This file was generated with po4a. Translate the source file.
%
\documentclass[11pt,a4paper,sans]{moderncv}
\moderncvstyle{classic}

\usepackage[scale=0.80]{geometry}

\makeatletter

\moderncvcolor{blue}

\nopagenumbers{}

\firstname{Lluís} \familyname{Ramon Callao}
\title{Data Scientist}

\address{Melcior de Palau 25}{Barcelona 08028} \mobile{609236313}
\email{lluisramon.tgn@gmail.com}
\social[linkedin][www.linkedin.com/in/lluisramon]{Lluís Ramon}
\social[github][www.github.com/LluisRamon]{LluisRamon}
\social[twitter][https://twitter.com/lluisr_tgn]{@LluisR\_Tgn} 

\photo[75pt][0.2pt]{../Fotos/Lluis_pedra_dni.jpg}


\begin{document}

\makecvtitle

\section{Educación}

\cventry{2013 - 2016}{Máster en Estadística e Investigación
Operativa}{Universitat Politècnica de Catalunya - Universitat de
Barcelona}{Barcelona}{}{Master Thesis: Using tree-based models to predict
drunk driving at Police preventive checkpoints.}

\cventry{2002 - 2007}{Licenciatura en Matemáticas}{Universitat Politècnica
de Catalunya}{Barcelona}{}{}

\cventry{2007}{Becario del programa Socrates-Erasmus}{Université Pierre et
Marie CURIE}{Paris}{}{}

\section{Experiencia}

\cventry{2016 - Presente}{Data Scientist}{Pagantis}{Barcelona}{}{Pagantis es
una Fintech dedicada al financiación al consumo con productos online,
automáticos y sin papeles. \newline Mi responsabilidad principal es mejorar
el Credit Risk Engine para reducir el riesgo y aumentar la aceptación de los
clientes. \newline Las principales tareas realizadas son:
\begin{itemize} \item Crear modelos de Machine Learning y nuevas reglas de
negocio
\begin{itemize} \item Definición, obtención y manipulación de datos, feature
engineering
\item Feature selection, model training y validación
\item Colaborar con el equipo de desarrollo para poner el modelo en
producción
\item Seguimiento y validación del modelo cuando ya está en producción
\end{itemize}
\item Optimización del porfolio para mejorar el trade-off entre aceptación y
riesgo
\item Diseñar un nuevo flujo de returnings para mejorar los KPI asociados
\end{itemize} }

\cventry{2007 - 2016}{Técnico en seguridad vial y movilidad}{Servei Català
de Trànsit, Generalitat de Catalunya}{Barcelona}{}{
El Servei Català de Trànsit es una agencia gubernamental encargada de la
seguridad vial y la movilidad en Catalunya. \newline
Mi responsabilidad principal era el seguimiento de los KPI de seguridad vial
y de movilidad en Catalunya \newline
Las principales tareas realizadas fueron:
\begin{itemize}
\item Automatizar el análisis de datos, mapas y reporting con librerias de R
internas
\item Validación de datos de accidentes, controles alcoholemia, radar,
espiras, sanciones.
\item Diseño de estudios, recolección de datos y su posterior análisis
\item Definición de los principales KPI para el correcto seguimiento de la
seguridad vial
\item Colaborar en la definición del plan de seguridad vial de Catalunya
\item Participar en proyectos europeos y en propuestas a la Comisión Europea
\item Enlace con proveedores, realización de pliegos y seguimiento de los
trabajos realizados por consultoras externas
\end{itemize}}

\cventry{Jun. 2015}{Curso de introducción al lenguaje de programación
R}{Datosphera}{}{Barcelona}{Formación de 16 horas al equipo técnico de
Datosphera en el lenguaje de programación R. Se adaptó el curso a las
necesidades de la empresa y a resolver sus problemas con los datos.}

\pagebreak

\section{Idiomas}

\cvitem{Catalan}{Competencia bilingüe o nativa} \cvitem{Español}{Competencia
bilingüe o nativa} \cvitem{Inglés}{Competencia profesional completa}
\cvitem{Francés}{Competencia básica limitada}

\section{Aptitudes y Conocimientos}

\cvitem{Programación}{SQL, MongoDB, Python, TEX, Git.} \cvitem{Data
Analysis}{R, Matlab, SPSS, SAS.} \cvitem{OS}{Windows, Mac, Linux.}
\cvitem{BI}{Microstrategy, Tableau.}

\section{Cursos}

\cventry{Apr. 2012}{Programming a Robotic Car}{Udacity}{Online}{}{}
\cventry{Mar. 2011}{Diseño y tratamiento de encuestas para el
muestreo}{Institut d'Estadística de Catalunya}{Barcelona}{}{}
\cventry{Sep. 2010}{Curso de contratos del sector público}{Escola
d'Administració Pública de Catalunya}{Online}{}{} \cventry{Sep. 2009}{Agata
GIS}{Servei Català de Trànsit}{Barcelona}{}{} \cventry{May
2009}{Microstrategy Desktop}{Microstrategy Business
Intelligence}{Barcelona}{}{} \cventry{Apr. 2008}{Aimsun Foundation
Course}{Transport Simulation Systems}{Barcelona}{}{}
\cventry{Feb. 2008}{Certificado de Aptitud Pedagógica}{Universitat
Politècnica de Catalunya}{Barcelona}{}{}

\section{Publicaciones}

\cventry{Oct. 2016}{\httplink[Drinking patterns and drunk-driving behavior
in Catalonia, Spain: A comparative
study]{https://www.sciencedirect.com/science/article/abs/pii/S1369847816303709}}{Manuela
Alcañiz, Miguel Santolino, Lluís Ramon}{Transportation Research Part F:
Traffic Psychology and Behavior}{}{}

\cventry{Apr. 2014}{\httplink[Prevalence of alcohol-impaired drivers based
on random breath tests in a roadside survey in Catalonia
(Spain)]{www.sciencedirect.com/science/article/pii/S0001457514000037}}{Manuela
Alcañiz, Montserrat Guillén, Miguel Santolino, Daniel Sánchez-Moscona, Oscar
Llatje, Lluís Ramon}{Accident Analysis \& Prevention}{}{}

\section{Presentaciones}

\cventry{Jan. 2016}{\link[Setting Rstudio Server using Amazon Web
Services]{http://lluisramon.github.io/RStudio-Server-AWS/Presentacion.html}}{Lluís
Ramon}{Barcelona R user's group}{Barcelona}{}{}
\cventry{Nov. 2015}{\link[Step by step introduction to
R]{http://lluisramon.github.io/intro_R_paso_a_paso/introducion_paso_a_paso.html}}{Lluís
Ramon}{Barcelona R user's group}{Barcelona}{}{}
\cventry{Jul. 2015}{\link[Package development
tutorial]{http://lluisramon.github.io/Taller_packages_R/Taller_packages_R}}{Lluís
Ramon}{Barcelona R user's group}{Barcelona}{}{}
\cventry{Feb. 2015}{\link[Automatic Reporting with
rmarkdown]{http://lluisramon.github.io/rmarkdown-talk/Presentacio}}{Carlos
Bort \& Lluís Ramon}{Barcelona R user's group}{Barcelona}{}{}
\cventry{Apr. 2014}{\link[Introduction to maps with
R]{http://lluisramon.github.io/Introduccio-mapes-amb-R/Intro_mapes_amb_R.html}}{Lluís
Ramon}{Barcelona R user's group}{Barcelona}{}{}
\cventry{Feb. 2014}{\link[Easy Dates and Times in
R]{http://rpubs.com/Lluis_Ramon/Easy_Date_Times_R}}{Lluís Ramon}{Barcelona R
user's group}{Barcelona}{}{} \cventry{Apr. 2013}{Easy package development
with RStudio}{Lluís Ramon}{Barcelona R user's group}{Barcelona}{}{}
\cventry{Nov. 2012}{\link[A practical introduction to ggplot2 and
ggmap]{http://rpubs.com/Lluis_Ramon/Prestantacion_ggplot2_ggmap}}{Andreu
Vall, Roger Borras, Lluís Ramon}{IV Spanish R User
Conference}{Barcelona}{}{} \cventry{Oct. 2012}{\link[Introduction to maps
with R]{http://rpubs.com/Lluis_Ramon/Presentacio_mapes_amb_R}}{Lluís
Ramon}{Barcelona R user's group}{Barcelona}{}{}
\cventry{Jun. 2012}{Introduction to R}{Lluís Ramon}{Barcelona R user's
group}{Barcelona}{}{} \cventry{Feb. 2012}{Data transformations with R}{Lluís
Ramon}{Barcelona R user's group}{Barcelona}{}{} \cventry{Oct. 2009}{SIDAT: A
comprehensive system for collecting road accidents data}{Lluís Ramon}{1st
Meeting on Road Safety for Cities}{Gijón}{}{}

\section{Premios}

\cventry{Oct. 2018}{eDreams ODIGEO Datathon - Second Best Accuracy
Model}{Predecir el tipo de alojamiento (hotel, albergue, pensión,
apartamento) utilizando varias imágenes y etiquetas de Google Cloud
Vision. Equipo con David Massip y Jordi Puigdellivol}{}{}{}{}

\cventry{Oct. 2016}{Social Point Monster Legends Datathon - Best Business
Insights Track}{Mejor Business Insights Track y tercer mejor sistema de
recomendaciones sobre datos de un videojuego. Equipo con Jordi Zamora,
Carlos Bort y Jordi Puigdellivol}{}{}{}{}

\cventry{May 2016}{\link[Accenture Datathon 2016 - Best Accuracy Track
Model]{https://www.accenture.com/es-es/careers/analytics-datathon}}{Predicción
de accidentes de tráfico en Barcelona a partir de informes de accidentes de
la Guardia Urbana. Equipo con Carlos Bort, Jordi Puigdellivol y Jordi
Zamora}{}{}{}{}

\cventry{Nov. 2015}{\link[Social Point Dragon City Datathon - Best Accuracy
Track Model]{http://bcnanalytics.com/hackathon/}}{Mejor modelo predictivo
para detectar jugadores que se darán de baja a un videojuego. Equipo con
Carlos Bort, Jordi Puigdellivol y Jordi Zamora}{}{}{}{}

\end{document}
