\documentclass[11pt,a4paper,sans]{moderncv}
\moderncvstyle{classic}

%% Adjust the page margins
%% \usepackage[scale=0.85]{geometry}
\usepackage[scale=0.80]{geometry}

%% Character encoding
\usepackage[utf8]{inputenc}

\makeatletter

% Add linkedIn logo 
% Code based on http://tex.stackexchange.com/questions/110342/moderncv-non-marvosym-symbols
% Code based on http://fernandomayer.github.io/blog/2013/03/31/moderncv-with-github-and-linkedin-logos/
\newcommand{\linkedin}[1]{\def\@linkedin{#1}}
\newcommand\linkedinsymbol{\raisebox{-1pt}{\includegraphics[height=9pt]{Fotos/linkedin}}\ }

% Add a linkedIn line from makecvtitle code
\renewcommand*{\makecvtitle}{%
  % recompute lengths (in case we are switching from letter to resume, or vice versa)
  \recomputecvlengths%
  % optional detailed information box
  \newbox{\makecvtitledetailsbox}%
  \savebox{\makecvtitledetailsbox}{%
    \addressfont\color{color2}%
    \begin{tabular}[b]{@{}r@{}}%
      \ifthenelse{\isundefined{\@addressstreet}}{}{\makenewline\addresssymbol\@addressstreet%
        \ifthenelse{\equal{\@addresscity}{}}{}{\makenewline\@addresscity}% if \addresstreet is defined, \addresscity and addresscountry will always be defined but could be empty
        \ifthenelse{\equal{\@addresscountry}{}}{}{\makenewline\@addresscountry}}%
      \ifthenelse{\isundefined{\@mobile}}{}{\makenewline\mobilesymbol\@mobile}%
      \ifthenelse{\isundefined{\@phone}}{}{\makenewline\phonesymbol\@phone}%
      \ifthenelse{\isundefined{\@fax}}{}{\makenewline\faxsymbol\@fax}%
      \ifthenelse{\isundefined{\@email}}{}{\makenewline\emailsymbol\emaillink{\@email}}%
      \ifthenelse{\isundefined{\@homepage}}{}{\makenewline\homepagesymbol\httplink{\@homepage}}%
      \ifthenelse{\isundefined{\@linkedin}}{}{\makenewline\linkedinsymbol{\@linkedin}}%
      \ifthenelse{\isundefined{\@extrainfo}}{}{\makenewline\@extrainfo}%
    \end{tabular}
  }%
  % optional photo (pre-rendering)
  \newbox{\makecvtitlepicturebox}%
  \savebox{\makecvtitlepicturebox}{%
    \ifthenelse{\isundefined{\@photo}}%
    {}%
    {%
      \hspace*{\separatorcolumnwidth}%
      \color{color1}%
      \setlength{\fboxrule}{\@photoframewidth}%
      \ifdim\@photoframewidth=0pt%
        \setlength{\fboxsep}{0pt}\fi%
      \framebox{\includegraphics[width=\@photowidth]{\@photo}}}}%
  % name and title
  \newlength{\makecvtitledetailswidth}\settowidth{\makecvtitledetailswidth}{\usebox{\makecvtitledetailsbox}}%
  \newlength{\makecvtitlepicturewidth}\settowidth{\makecvtitlepicturewidth}{\usebox{\makecvtitlepicturebox}}%
  \ifthenelse{\lengthtest{\makecvtitlenamewidth=0pt}}% check for dummy value (equivalent to \ifdim\makecvtitlenamewidth=0pt)
    {\setlength{\makecvtitlenamewidth}{\textwidth-\makecvtitledetailswidth-\makecvtitlepicturewidth}}%
    {}%
  \begin{minipage}[b]{\makecvtitlenamewidth}%
    \namestyle{\@firstname\ \@familyname}%
    \ifthenelse{\equal{\@title}{}}{}{\\[1.25em]\titlestyle{\@title}}%
  \end{minipage}%
  \hfill%
  % detailed information
  \llap{\usebox{\makecvtitledetailsbox}}% \llap is used to suppress the width of the box, allowing overlap if the value of makecvtitlenamewidth is forced
  % optional photo (rendering)
  \usebox{\makecvtitlepicturebox}\\[2.5em]%
  % optional quote
  \ifthenelse{\isundefined{\@quote}}%
    {}%
    {{\centering\begin{minipage}{\quotewidth}\centering\quotestyle{\@quote}\end{minipage}\\[2.5em]}}%
  \par}% to avoid weird spacing bug at the first section if no blank line is left after \makecvtitle

\moderncvcolor{blue}


\renewcommand{\familydefault}{\sfdefault} 

\nopagenumbers{}

\firstname{Lluís}
\familyname{Ramon Callao}
\title{Currículum vítae}
%% \title{Data Scientist}

\address{Tirso de molina 17}{Barcelona 08028}
\mobile{609236313}
\email{lluisramon.tgn@gmail.com}
\linkedin{\httplink[Lluís Ramon]{www.linkedin.com/in/lluisramon}} 

\photo[75pt][0.2pt]{Fotos/CamisaParc2.jpg}

%% \quote{Data Scientist)}

% Structure and concepts taken from Xevi Pujol's CV http://xvpujol.com/
\begin{document}

\makecvtitle

\section{Education}
\cventry{2002 - 2007}{Degree in Mathematics}{Universitat Politècnica de Catalunya}{Barcelona}{}{Degree that included a wide range of courses in several areas of knowledge:
\begin{itemize}
\item Pure and Applied Mathematics (Algebra, Geometry, Mathematical Modelling, Real Analysis, Differential equations). 
\item Statistics (Operations Research, Statistical Inference).
\item Computing (Algorithmics, Programming, Numerical Analysis).
\end{itemize}}

\cventry{Feb. - Jun. 2007}{Socrates-Erasmus Exchange Program}{Université Pierre et Marie CURIE}{Paris}{}{
\begin{itemize}
\item Modules taught entirely in French.
\item Attended Mathematical optimization, Epidemiology and Functional analysis.
\end{itemize}}

\section{Experience}
\cventry{Dec. 2007 - Present}{Road Safety and Traffic Management Engineer}{Servei Català de Trànsit, Generalitat de Catalunya}{Barcelona}{}{
\begin{itemize}
\item Statistical analysis and reporting in the following fields: 
  \begin{itemize}
  \item Road safety
  \item Breath alcohol content enforcement
  \item Speed 
  \item Mobility
  \item Traffic congestion 
  \item Relationship between mobility and accidents
  \end{itemize}
\item Preparation of contract specifications for hiring external studies. Project management and final document supervision.
\item Participation in European projects and proposals to the European Commission.
\item Studying and proposing policies in the frame of the Catalan road safety plan. 
\item Data analysis and database management of the following DBs:
  \begin{itemize}
  \item Road accidents (24.000 entries/year on 180 variables)
  \item Road mobility (4.000.000 entries/year on 20 variables)
  \item Breath alcohol content enforcement (600.000 entries/year on 15 variables)
  \item Traffic congestions (24.000 entries/year on 25 variables)
  \end{itemize}
\item Development of a package in the R programming language to automate the following tasks:
  \begin{itemize}
  \item Connection and data collection from DataWareHouse
  \item Specialized functions for road safety and mobility data processing
  \item Geographical representation of information and map generation
  \item Mathematical-statistical model to treat mobility data
  \item Mathematical-statistical model to treat speed data
  \item Automatic report generation
  \end{itemize}
\end{itemize}}

\pagebreak

\section{Languages}

%% Based on ILR Scale http://en.wikipedia.org/wiki/ILR_scale
\cvitem{Catalan}{Native proficiency }
\cvitem{Spanish}{Native proficiency }
\cvitem{English}{Full professional proficiency}
\cvitem{French}{Elementary proficiency}

\section{IT Skills}

\cvitem{Programming}{SQL, Python, TEX, Git.}
\cvitem{Data Analysis}{R, Matlab, SPSS.}
\cvitem{OS}{Windows, Mac, Linux.}
\cvitem{BI}{Microstrategy.}

\section{Courses}

\cventry{Apr. 2012}{Programming a Robotic Car}{Udacity}{Online}{}{}
\cventry{Mar. 2011}{Survey sampling design and analysis}{Institut d'Estadística de Catalunya}{Barcelona}{}{}
\cventry{Sep. 2010}{Public Contracts Specification}{Escola d'Administració Pública de Catalunya}{Online}{}{}
\cventry{Sep. 2009}{Agata GIS}{Servei Català de Trànsit}{Barcelona}{}{}
\cventry{May 2009}{Microstrategy Desktop}{Microstrategy Business Intelligence}{Barcelona}{}{}
\cventry{Apr. 2008}{Aimsun Foundation Course}{Transport Simulation Systems}{Barcelona}{}{}
\cventry{Feb. 2008}{Certificate of Pedagogical Aptitude}{Universitat Politècnica de Catalunya}{Barcelona}{}{}

\section{Publications}

\cventry{Abr. 2013}{\httplink[Relationships flow/speed (congestion) on the arterial road network to Barcelona]{www.aecarretera.com/servicios/publicaciones/revista-carreteras/articulos-publicados/226-revista-carreteras-n-188/1946-relaciones-intensidades-velocidades-congestion-en-los-accesos-a-barcelona}
}{Òscar Llatje, Enric Homedes, Ferran Muñoz, Lluís Ramon}{Revista Carreteras}{Asociación Española de la Carretera}{}

\section{Presentations}

\cventry{Apr. 2013}{Easy package development with RStudio}{Lluís Ramon}{Barcelona R user's group}{Barcelona}{}{}
\cventry{Nov. 2012}{\link[A practical introduction to ggplot2 and ggmap]{http://rpubs.com/Lluis_Ramon/Prestantacion_ggplot2_ggmap}}{Andreu Vall, Roger Borras, Lluís Ramon}{IV Spanish R User Conference}{Barcelona}{}{} 
\cventry{Oct. 2012}{\link[Introduction to maps with R]{http://rpubs.com/Lluis_Ramon/Presentacio_mapes_amb_R}}{Lluís Ramon}{Barcelona R user's group}{Barcelona}{}{}
\cventry{Jun. 2012}{Introduction to R}{Lluís Ramon}{Barcelona R user's group}{Barcelona}{}{}
\cventry{Feb. 2012}{Data transformations with R}{Lluís Ramon}{Barcelona R user's group}{Barcelona}{}{}

\end{document}